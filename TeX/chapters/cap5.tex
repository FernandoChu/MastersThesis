\documentclass[../main.tex]{subfiles}
\begin{document}
Hemos introducido la Teor\'ia Homot\'opica de Tipos, en donde el concepto principal es de tipos y elementos de tipos.
Hemos visto que esta teor\'ia puede interpretarse desde tres perspectivas distintas, la l\'ogica, la categ\'orica y la homot\'opica.

Desde la perspectiva l\'ogica, los tipos representan proposiciones y los elementos representan las pruebas de estas.
Desde la perspectiva categ\'orica, los tipos son $\infty$-grupoides, siendo los objetos los elementos del tipo, y los morfismos las igualdades entre ellos.
Desde la perspectiva homot\'opica, los tipos son tipos de homotop\'ia de espacios topol\'ogicos, y las funciones entre tipos corresponden a funciones continuas.

Hemos visto que, utilizando un número muy limitado de reglas y axiomas, esta teor\'ia permite formalizar de una forma m\'as abstracta conceptos comunes de la matem\'atica como lo son los grupos, caminos, homotop\'ia, entre otros.

En particular, vemos que podemos definir CW complejos de una forma m\'as elegante, utilizando Tipos Inductivos Superiores, y podemos manejarlos de una manera similar a la usual, lo que nos ha permitido demostrar que $\pi_1(\Sn^1)=\Z$, a modo de ejemplo.

HoTT todav\'ia es una rama nueva en desarrollo, y existen m\'ultiples grupos que est\'an traduciendo la matem\'atica cl\'asica a este nuevo lenguaje.
Creemos que en el futuro HoTT, o una variante similar a esta, ser\'a la teor\'ia principal sobre la cual se desarrollar\'a las matem\'aticas, y la teor\'ia de conjuntos ya no ser\'a la herramienta principal para formalizar conceptos.

Finalizamos exhortando al lector a indagar m\'as sobre HoTT, y a contribuir en algunas de las organizaciones que promueven su adopci\'on, como lo son \textit{Agda-Unimath} y \textit{1Lab}.
\end{document}
